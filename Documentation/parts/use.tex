\chapter{How to use it?}

Once the installation steps are done, using the client is quite easy.
You simply need to launch the client\_shell in a terminal using the command:
\begin{lstlisting}
	###: ./client_shell
\end{lstlisting}

It will launch a shell-like interface.
In this interface, you can input some commands to control the client.

We have two type of command:
\begin{itemize}
	\item built in functions: functions written in c to control the system
	\item classic shell command
\end{itemize}

We are mainly interested in the built in functions.
The available commands are:
\begin{itemize}
	\item help: list the available built in functions
	\item exit: close the client\_shell
	\item launch: launch a client
	\item findServers: update the list of available servers
	\item displayServers: display the available servers
	\item close: close a client
	\item list: list the clients currently running 
	\item log: start or stop logging the data for one client
\end{itemize}

If you type a function but do not put the right parameters, the shell will display a short message
explaining how to use the function properly.

Here is an example of process you want to do when starting the client\_shell for the first time:
First we want to find the available servers. So we type:
\begin{lstlisting}
	displayServers
\end{lstlisting}
We get a list of the available servers. Each server has a number in front of it. We can use this number to start a client who will communicate with this server.
\begin{lstlisting}
	launch 0
\end{lstlisting}

A new window pops up. This is the client we just created.
We can start another client on another server.
\begin{lstlisting}
	launch 1
\end{lstlisting}


Now we want to list the clients running. So we type:
\begin{lstlisting}
	list
\end{lstlisting}
We get a list of the clients running. Each client has a number in front of it. We can use this number to send messages to the client.


For instance, we may want to log the data of the first client in a file called "`Log;\_UHF"'.
\begin{lstlisting}
	log 0 Log_UHF
\end{lstlisting}
The client with the index 0 will start to log the data in the file Log\_UHF.

After a while, we think we have enough data and we want to stop logging them.
\begin{lstlisting}
	log 0 stop
\end{lstlisting}

With the same command log wa can stop it. However, it means stop can't be the filename.

Finally we want to shutdown all the clients.
\begin{lstlisting}
	close all
\end{lstlisting}
This will close all the clients and their connections. If we wanted to shutdown only one client, we just have to replace "`all"' by the number of the client.






