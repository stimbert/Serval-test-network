\chapter{Resources used}

In this part, you will find every useful references I used to create the test network. If you want to have a better understanding on the functioning of the solution, you can take a look at them

\section{Network}
Source for relayd: \url{https://wiki.openwrt.org/doc/recipes/relayclient}

DNS and DHCP: \url{https://wiki.openwrt.org/doc/uci/dhcp}

\section{server}


function poll: \url{https://www.ibm.com/support/knowledgecenter/en/ssw\_i5\_54/rzab6/poll.htm}

cross-compilation: \url{https://manoftoday.wordpress.com/2007/10/11/writing-and-compiling-a-simple-program-for-openwrt/}
cross-compilation: \url{https://github.com/airplug/airplug/wiki/Cross-compiling-for-OpenWRT}


\section{client}

signal handling: \url{https://stackoverflow.com/questions/5546223/signals-received-by-bash-when-terminal-is-closed}


\section{client\_shell}

tutorial to write a shell in C: \url{https://brennan.io/2015/01/16/write-a-shell-in-c/}
customise xterm: \url{https://scarygliders.net/2011/12/01/customize-xterm-the-original-and-best-terminal/}



\section{Phone}

usb over ip: \url{https://wiki.openwrt.org/doc/howto/usb.iptunnel}
scrcpy: \url{https://github.com/Genymobile/scrcpy}
FIFO: \url{https://www.tldp.org/LDP/lpg/node11.html}



\section{Other useful references}
Boot structure of openwrt: \url{https://medium.com/openwrt-iot/lede-openwrt-boot-structure-e689c4ddea91}
Scheduling tasks with cron on openwrt: \url{https://medium.com/openwrt-iot/openwrt-scheduling-tasks-6e19d507ae45}