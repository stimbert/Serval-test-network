\chapter{Conclusion}

Overall, I am satisfied with what I was able to do in less than 2 months. I have developed a first prototype with the core features.
I think that the solution is easy to use even for someone not very familiar with the Serval project and Linux. Obviously, the user still need to know some basic about Linux and networking to install the solution. But I think that most person working on the Serval project will have some knowledge on that field and should therefore have no problem to install and use the test network. To help future users, I wrote a detailed documentation for the installation and the first use.

This project taught me a lot. I already have some knowledge on c coding, networking and Linux but I haven't trained these skills in more than a year. This project gave me some experience in this field and helped me improve. Through this project and my time spent in Australia, I have improved my English. For me it is the first time I am writing a report of this size in English. Furthermore, all the interaction with the team have made me more confident in my English skills.
 
Some mistakes were made during this project. The main mistake was the management of the different task. I feel I have lost more time than I should have on the mobile part. It was not one of the priority of the test network and I spent a lot of time on it. I could have used this time to implement the Wi-Fi part and have a better final result. I should have asked more for the help of my supervisor. I did ask some questions from time to time and his answers were really useful. But I feel like I should have asked more question and keep a better communication with my supervisor.

The project still lacks some features to be considered complete. They should be added in the future.
First, the test network should be done in real scale in Tonsley to see if it still works in the real case.

Then, the missing features should be developed and added. 
At this point we will have a real test network.

After that, improvements can be done to the solution. For instance, I think some part of the installation process could be made easier.
Another major improvement could be to replace the client\_shell by a graphical interface. A graphical interface will make the solution more instinctive to use. That was one of my wish when I started this project. I wanted to have a graphical interface for the user. However, I knew I would not have the time to create it. That's why I decided to opt for the shell-like solution.

Even if I made some mistakes and the test network is not complete yet, I am still very happy with the final result and everything I have learned.
